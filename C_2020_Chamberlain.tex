\documentclass[9pt]{beamer}
\usefonttheme{serif}
\usepackage{amssymb,amsfonts,amsmath,bbm}
\usepackage{subcaption,graphicx,hyperref}
\captionsetup{compatibility=false}

\setcounter{MaxMatrixCols}{30}
\usepackage{scalefnt}

\providecommand{\U}[1]{\protect\rule{.1in}{.1in}}
\newenvironment{stepenumerate}{\begin{enumerate}[<+->]}{\end{enumerate}}
\newenvironment{stepitemize}{\begin{itemize}[<+->]}{\end{itemize} }
\newenvironment{stepenumeratewithalert}{\begin{enumerate}[<+-| alert@+>]}{\end{enumerate}}
\newenvironment{stepitemizewithalert}{\begin{itemize}[<+-| alert@+>]}{\end{itemize} }
\usetheme{Boadilla}

\AtBeginSection[]
{
\begin{frame}<beamer>
\frametitle{Outline}
\tableofcontents[currentsection]
\end{frame}
}
\AtBeginSubsection[]
{
\begin{frame}[plain]{Outline}
\tableofcontents[currentsection,currentsubsection]
\end{frame}
}
\newcommand\independent{\protect\mathpalette{\protect\independenT}{\perp}}
\def\independenT#1#2{\mathrel{\rlap{$#1#2$}\mkern2mu{#1#2}}}
\newcommand{\I}{{\rm 1\hspace*{-0.4ex}\rule{0.1ex}{1.52ex}\hspace*{0.2ex}}}
\setbeamertemplate{navigation symbols}{}
\setbeamertemplate{footline}{}

% Define some colors:
\definecolor{DarkFern}{HTML}{407428}
\definecolor{DarkCharcoal}{HTML}{4D4944}
\colorlet{Fern}{DarkFern!85!white}
\colorlet{Charcoal}{DarkCharcoal!85!white}
\colorlet{LightCharcoal}{Charcoal!50!white}
\colorlet{AlertColor}{red}
\colorlet{DarkRed}{red!70!black}
%\colorlet{DarkBlue}{blue!80!black}
\definecolor{DarkBlue}{rgb}{0,0,0.9}
\colorlet{DarkGreen}{green!70!black}
\definecolor{shadecolor}{rgb}{0.9,0.9,0.8}
\definecolor{shadecolor2}{rgb}{0.7,0.7,0.7}

\newcommand{\light}[1]{\textcolor{shadecolor2}{#1}}
\newcommand{\white}[1]{\textcolor{white}{#1}}

% Blackboard shortcuts.
\newcommand{\E}{\mathbb{E}}
\newcommand{\V}{\mathbb{V}}
\renewcommand{\P}{\mathbb{P}}
\renewcommand{\I}{\mathbbm{1}}

% RD shortcuts
\renewcommand{\c}{c}
\newcommand{\W}{\mathcal{W}}

% Bold fonts.
\newcommand{\bY}{\mathbf{Y}}
\newcommand{\bT}{\mathbf{T}}
\newcommand{\bX}{\mathbf{X}}
\newcommand{\bZ}{\mathbf{Z}}
\newcommand{\bz}{\mathbf{z}}
\newcommand{\be}{\mathbf{e}}
\newcommand{\br}{\mathbf{r}}
\newcommand{\bP}{\mathbf{P}}
\newcommand{\bs}{\mathbf{s}}
\newcommand{\bS}{\mathbf{S}}
\newcommand{\bF}{\mathbf{F}}
\newcommand{\bR}{\mathbf{R}}
\newcommand{\bK}{\mathbf{K}}
\newcommand{\bI}{\mathbf{I}}

\newcommand{\bbeta}{\boldsymbol{\beta}}
\newcommand{\bgamma}{\boldsymbol{\gamma}}
\newcommand{\btau}{\boldsymbol{\tau}}
\newcommand{\btheta}{\boldsymbol{\theta}}
\newcommand{\bdelta}{\boldsymbol{\delta}}
\newcommand{\bmu}{\boldsymbol{\mu}}
\newcommand{\bvarphi}{\boldsymbol{\varphi}}
\newcommand{\biota}{\boldsymbol{\iota}}

%% path to where some of the figures are
\newcommand{\pathfig}{figures}

\renewcommand{\thefootnote}{}

%%%%%%%%%%%%%%%%%%%%%%%%%%%
%
% Set arrows
%
%%%%%%%%%%%%%%%%%%%%%%%%%%%%%
\usepackage{tikz}
\usetikzlibrary{shapes.arrows}


\tikzset{
	myarrow/.style={
		draw,
		fill=orange,
		single arrow,
		minimum height=3.5ex,
		single arrow head extend=1ex
	}
}
\newcommand{\arrowup}{%
	\tikz [baseline=-0.5ex]{\node [myarrow,rotate=90] {};}
}
\newcommand{\arrowdown}{%
	\tikz [baseline=-1ex]{\node [myarrow,rotate=-90] {};}
}
%----------------------------------------------


\begin{document}

\title[]{Tutorial on Regression Discontinuity Designs
	     \thanks{Prepared for \href{https://www.chamberlainseminar.org/}{The Gary Chamberlain Online Seminar in Econometrics}.}
	     \thanks{Complementary materials available at \url{https://rdpackages.github.io/}.}\medskip}
\author{Matias D. Cattaneo\medskip \\
	    Princeton University\medskip }
\maketitle



%%%%%%%%%%%%%%%%%%%%%%%%%%%%%%%%%%%%%%%%%%%%%%%%%%%%%%%%%%%%
\section{Designs and Frameworks}
%%%%%%%%%%%%%%%%%%%%%%%%%%%%%%%%%%%%%%%%%%%%%%%%%%%%%%%%%%%%

\begin{frame}\frametitle{Causal Inference \& Program Evaluation}
	\begin{itemize}
		\item Main goal: learn about treatment effect of policy or intervention.\bigskip
		
		\item If treatment randomization available, easy to estimate treatment effects.\bigskip
		
		\item If treatment randomization not available, turn to observational studies.\medskip
		
		\begin{itemize}
			\item Selection on Observables, Instrumental Variables, Selection on Unobservables, etc.\bigskip
		\end{itemize}
		
		\item \textbf{Regression Discontinuity (RD) designs}.\medskip
		
		\begin{itemize}
			\item Simple and objective. Requires little information, if design available.\medskip
			
			\item Might be viewed as a ``local'' randomized trial.\medskip
			
			\item Easy to ``falsify'' and easy to interpret.\medskip
			
			\item \emph{Careful}: very local!
		\end{itemize}
	\end{itemize}
\end{frame}

\begin{frame}\frametitle{RD Designs: Building Block}
	\begin{itemize}
		\item \textbf{Triplet}: \textit{score}, \textit{cutoff}, \textit{treatment}.\medskip 
		\begin{itemize}
			\item Units receive a score.\medskip 
			
			\item A treatment is assigned based on the score and a \textit{known} cutoff.\medskip
			
			\item The treatment\smallskip
			\begin{itemize}
				\item[] is offered to units whose score is greater than the cutoff.\smallskip 
				\item[] is withheld from units whose score is less than the cutoff.\medskip
			\end{itemize}
		\end{itemize}

	\begin{figure}[h]
		\centering\includegraphics[scale=0.30]{\pathfig/Fig1-AssProbSharp.pdf}
	\end{figure}

		\item Under assumptions, the abrupt change in the probability of treatment assignment allows us to learn about causal treatment effects.\medskip 
	\end{itemize}
\end{frame}

\begin{frame}\frametitle{RD Designs: Taxonomy}
	\begin{itemize}
		\item \textbf{Frameworks}.\medskip
		\begin{itemize}
			\item Identification: Continuity/Extrapolation,	Local Randomization.\medskip
			\item Score: Continuous, Many Repeated, Few Repeated.\bigskip
		\end{itemize}
		
		\item \textbf{Settings}.\medskip
		\begin{itemize}
			\item Sharp, Fuzzy, Kink, Kink Fuzzy.\medskip
			
			\item Multiple Cutoff, Multiple Scores, Geographic RD.\medskip
			
			\item Dynamic, Continuous Treatments, Time, etc.\bigskip
		\end{itemize}
	
		\item \textbf{Parameters of Interests}.\medskip
		\begin{itemize}
			\item Average Effects, Quantile/Distributional Effects, Partial Effects.\medskip
			\item Heterogeneity, Covariate-Adjustment, Differences, Time.\medskip
			\item Extrapolation.
		\end{itemize}
	\end{itemize}
\end{frame}

\begin{frame}[t]\frametitle{RCTs vs. (Sharp) RD Designs}
	\bigskip
	\begin{minipage}{\textwidth}
	\begin{itemize}
		\item \textbf{Notation}: $(Y_i(0),Y_i(1),X_i)$, $i=1,2,\dots,n$.\bigskip
		
		\item \textbf{Treatment}: $T_i\in\{0,1\}$,\qquad$T_i$ independent of $(Y_i(0),Y_i(1),X_i)$.\bigskip
		
		\item \textbf{Data}: $(Y_{i},T_{i},X_{i})$, $i=1,2,\dots,n$, with
		\[
		Y_{i}=\left\{
		\begin{tabular}
			[c]{lll}%
			$Y_{i}(0)$ &  & if $T_{i}=0\smallskip$\\
			$Y_{i}(1)$ &  & if $T_{i}=1$
		\end{tabular}
		\right.
		\]
	\end{itemize}\bigskip
	\end{minipage}
	\begin{minipage}{\textwidth}
	\begin{itemize}
		\item \textbf{Average Treatment Effect}:
		\[\tau_{\mathtt{ATE}}=\E[Y_{i}(1)-Y_{i}(0)]=\E[Y_{i}|T=1]-\E[Y_{i}|T=0]\]
		
	\end{itemize}
	\end{minipage}	
\end{frame}

\begin{frame}[t]\frametitle{RCTs vs. (Sharp) RD Designs}
	\bigskip
	\begin{minipage}{\textwidth}
	\begin{itemize}
		\item \textbf{Notation}: $(Y_{i}(0),Y_{i}(1),X_{i})$, $i=1,2,\dots,n$, \qquad$X_{i}$ score.\bigskip
		
		\item \textbf{Treatment}: $T_{i}\in\{0,1\}$,\qquad$T_{i}=\I(X_{i}\geq \c)$,\qquad $\c$ cutoff.\bigskip
		
		\item \textbf{Data}: $(Y_{i},T_{i},X_{i})$, $i=1,2,\dots,n$, with
		\[
		Y_{i}=\left\{
		\begin{tabular}
			[c]{lll}%
			$Y_{i}(0)$ &  & if $T_{i}=0\smallskip$\\
			$Y_{i}(1)$ &  & if $T_{i}=1$%
		\end{tabular}
		\right.
		\]
	\end{itemize}\bigskip
	\end{minipage}
	\begin{minipage}{\textwidth}
	\begin{itemize}
		\item \textbf{Average Treatment Effect at the cutoff} (Continuity-based):
		\[\tau_{\mathtt{SRD}} = \E[Y_i(1)-Y_i(0)|X_i=\c] = \lim_{x\downarrow\c}\E[Y_i|X_i=x]-\lim_{x\uparrow\c}\E[Y_i|X_i=x]\]\smallskip
		
		\item \textbf{Average Treatment Effect in a neighborhood} (LR-based):
		\[\tau_{\mathtt{LR}} = \E[Y_i(1)-Y_i(0)|X_{i}\in\W] = \frac{1}{N_1} \sum_{X_i\in\W,T_i=1} Y_i - \frac{1}{N_0} \sum_{X_i\in\W,T_i=0} Y_i\]
	\end{itemize}
	\end{minipage}
\end{frame}

\begin{frame}
	\[\tau_\mathtt{SRD}
	  = \underset{\textcolor{black}{\text{Unobservable}}}{\underbrace{\E[Y_{i}(1)-Y_{i}(0)|X_i=\c]}} 
	  = \underset{\textcolor{red}{\text{Estimable}}}{\underbrace{\lim_{x\downarrow{\c}} \E[Y_i|X_i=x]}} 
	     - \underset{\textcolor{DarkBlue}{\text{Estimable}}}{\underbrace{\lim_{x\uparrow{\c}} \E[Y_i|X_i=x]}}
	\]
	\begin{figure}
		\vspace{-0.1in}
		\centering
		\includegraphics[scale=0.5]{\pathfig/Fig3-IdentSharp-step3.pdf}
	\end{figure}
\end{frame}

\begin{frame}
	\[ T_i \text{ independent of } (Y_i(0),Y_i(1)) \text{ for all } X_i\in\W=[\c-w,c+w] \]
	\begin{figure}
		\vspace{-0.1in}
		\centering
		\includegraphics[scale=0.5]{\pathfig/Fig11-RDLocRand.pdf}
	\end{figure}
\end{frame}

\begin{frame}\frametitle{Fuzzy RD Designs}
	\begin{itemize}
		\item \textbf{Imperfect compliance.}\medskip 
		\begin{itemize}
			\item probability of receiving treatment changes at $\c$, but not necessarily from 0 to 1.\bigskip
		\end{itemize}
		
		\item Canonical Parameter:
		\begin{align*}
			\tau_{\mathtt{FRD}}
			&=\frac{\E[(Y_{i}(1)-Y_{i}(0)(D_{i}(1)-D_{i}(0)))|X_{i}=\c]}{\E[D_{i}(1)|X_{i}=\c]-\E[D_{i}(0)|X_{i}=\c]}\\
			&=\frac{\lim_{x\downarrow\c}\E[Y_i|X_i=x]-\lim_{x\uparrow\c}\E[Y_i|X_i=x]}{\lim_{x\downarrow\c}\E[D_i|X_i=x]-\lim_{x\uparrow\c}\E[D_i|X_i=x]}
		\end{align*}
		where $Y_{i}(t) = Y_{i}(0)(1-D_{i}(t)) + Y_{i}(1)D_{i}(t)$ and $D_{i}(t) = D_{i}(0)(1-T_{i}) + D_{i}(1)T_{i}$.\bigskip
		
		\item Similarly for Local Randomization framework.\bigskip
		
		\item Different interpretations under different assumptions.\bigskip
	\end{itemize}
\end{frame}

\begin{frame}
	\begin{figure}[h]
		\begin{subfigure}[t]{0.47\textwidth}
			\centering
			\includegraphics[scale=0.30]{\pathfig/Fig1-AssProbSharp.pdf}
			\caption{Sharp RD}       
		\end{subfigure}
		\hspace{0.1in}
		\begin{subfigure}[t]{0.47\textwidth}
			\centering
			\includegraphics[scale=0.30]{\pathfig/Fig1-AssProbFuzzy.pdf}
			\caption{Fuzzy RD (one-sided compliance)}
		\end{subfigure}
	\end{figure}
\end{frame}

\begin{frame}\frametitle{(Sharp and Fuzzy) Kink RD Designs}
	\begin{itemize}
		\item Treatment assigned via continuous score formula, but slope changes discontinuously at ``kink'' point ($\c$).\bigskip
		\item SKRD Parameter:
		\[\tau_{\mathtt{KRD}}=\frac{\lim_{x\downarrow\c}\frac{d}{dx}\E[Y_{i}|X_{i}=x]-\lim_{x\uparrow\c}\frac{d}{dx}\E[Y_{i}|X_{i}=x]}
		                           {\lim_{x\downarrow\c}\frac{d}{dx}b(x)-\lim_{x\uparrow\c}\frac{d}{dx}b(x)}
		\]
		where $b(x)$ known function inducing ``kink''.\bigskip
		
		\item FKRD Parameter:
		\[\tau_{\mathtt{KRD}}=\frac{\lim_{x\downarrow\c}\frac{d}{dx}\E[Y_{i}|X_{i}=x]-\lim_{x\uparrow\c}\frac{d}{dx}\E[Y_{i}|X_{i}=x]}
		                           {\lim_{x\downarrow\c}\frac{d}{dx}\E[D_{i}|X_{i}=x]-\lim_{x\uparrow\c}\frac{d}{dx}\E[D_{i}|X_{i}=x]}
		\]
		\smallskip
		
		\item Different interpretation under different assumptions.
	\end{itemize}
\end{frame}

\begin{frame}
	\begin{figure}[h]
		\begin{subfigure}[t]{0.47\textwidth}
			\centering
			\includegraphics[scale=0.30]{\pathfig/Fig15-Kink-levels.pdf}
			\caption{Kink RD (levels)}       
		\end{subfigure}
		\hspace{0.1in}
		\begin{subfigure}[t]{0.47\textwidth}
			\centering
			\includegraphics[scale=0.30]{\pathfig/Fig15-Kink-derivatives.pdf}
			\caption{Kink RD (derivatives)}
		\end{subfigure}
	\end{figure}
\end{frame}

\begin{frame}\frametitle{Multi-cutoff, Multi-Score, Geographic RD Designs}
	\begin{itemize}
		\item \textbf{Multi-cutoff RD designs}.\medskip
		\begin{itemize}
			\item $C_{i}\in\mathcal{C}$ with $\mathcal{C}=\{c_{1},c_{2},\cdots,c_{J}\}$ or $\mathcal{C}=\mathcal{[}\underline{c},\overline{c}]$.\medskip
			\item Two strategies: normalize-and-pool ($\tilde{X}_{i}=X_{i}-C_{i}$), or cutoff-by-cutoff analysis.\medskip
			\item Different interpretation under different assumptions.\bigskip
		\end{itemize}
		
		\item \textbf{Multi-score RD designs}.\medskip
		\begin{itemize}
			\item $\mathbf{X}_{i}=(X_{1i},X_{2i},\dots,X_{di})'$ and $\mathbf{\c}=(\c_1,\c_2,\dots,c_d)'$.\medskip
			\item Can always be mapped back to Multi-cutoff RD designs.\medskip
			\item Leading special cases: Test scores, geography ($d=2$).\medskip
			\item Different interpretation under different assumptions.\bigskip
		\end{itemize}
		
		\item \textbf{Other RD-like designs}.\medskip
		\begin{itemize}
			\item RD in density and bunching designs.\medskip
			\item RD in time.\medskip
			\item Dynamic RD designs.\medskip
			\item etc.
		\end{itemize}
	\end{itemize}
\end{frame}

\begin{frame}
	\begin{figure}[h]
		\begin{subfigure}[t]{0.48\textwidth}
			\centering
			\includegraphics[scale=0.30]{\pathfig/Fig13-MultiCutoffRD.pdf}
			\caption{Multi-cutoff:\newline $\tau_\mathtt{SRD}(x,c)=\E[Y_i(1)-Y_i(0)|X_i=x,C_i=c]$}       
		\end{subfigure}
		\begin{subfigure}[t]{0.5\textwidth}
			\centering
			\includegraphics[scale=0.30]{\pathfig/Fig14-MultiScoreRD-Example.pdf}
			\caption{Multi-score:\newline $\tau_\mathtt{SRD}(x_1,x_2)=\E[Y_i(1)-Y_i(0)|X_{1i}=x_1,X_{2i}=x]$}
		\end{subfigure}
	\end{figure}
\end{frame}

\begin{frame}
	\begin{figure}
		\centering\includegraphics[scale=.3] {\pathfig/GeoRD-NJ-Display.pdf}
	\end{figure}
\end{frame}

\begin{frame}\frametitle{Highlights and Main Takeaways}
	\begin{itemize}
		\item RD designs exploit ``variation'' near the cutoff.\bigskip
		
		\item Causal effect is different (in general) than RCT.\bigskip

		\item No ``overlap'' (sharp) so extrapolation is unavoidable (local or global).\bigskip

		\item Graphical analysis is both very useful and very dangerous.\bigskip

		\item Need to work with data near cutoff $\Longrightarrow$ bandwidth or window selection.\bigskip
		
		\item There exist many design-specific falsification/validation methods.\bigskip
		
	\end{itemize}
\end{frame}

%%%%%%%%%%%%%%%%%%%%%%%%%%%%%%%%%%%%%%%%%%%%%%%%%%%%%%%%%%%%
\section{Estimation and Inference}
%%%%%%%%%%%%%%%%%%%%%%%%%%%%%%%%%%%%%%%%%%%%%%%%%%%%%%%%%%%%

\begin{frame}\frametitle{Empirical Illustration: Head Start (Ludwig and Miller, 2007,QJE)}
	\begin{itemize}
		\item \textbf{Problem}: impact of Head Start on Infant Mortality\bigskip
		
		\item \textbf{Data}:\medskip
		
		~\qquad$Y_{i}=$ child mortality 5 to 9 years old\medskip
		
		~\qquad$T_{i}=$ whether county received Head Start assistance\medskip
		
		~\qquad$X_{i}=$ 1960 poverty index\quad($\c=59.1984$)\medskip
		
		~\qquad$Z_{i}=$ see database.\bigskip
		
		\item \textbf{Potential outcomes}:\medskip
		
		~\qquad$Y_{i}(0)=$ child mortality if \textbf{had not received} Head
		Start\medskip
		
		~\qquad$Y_{i}(1)=$ child mortality if \textbf{had received} Head Start\bigskip
		
		\item \textbf{Causal Inference}:\medskip
		
		~\qquad$Y_{i}(0) \quad\neq\quad Y_{i}|T_{i}=0$\qquad\qquad and\qquad\qquad$Y_{i}(1) \quad\neq\quad Y_{i}|T_{i}=1$\bigskip
	\end{itemize}
\end{frame}

\begin{frame}\frametitle{RD Packages}
	\smallskip
	\begin{center}
		\url{https://rdpackages.github.io/}\medskip
	\end{center}
	
	\begin{itemize}
		\item \textbf{rdrobust}: estimation, inference and graphical presentation using local polynomials, partitioning, and spacings estimators.\medskip
		\begin{itemize}
			\item \texttt{rdrobust}, \texttt{rdbwselect}, \texttt{rdplot}.\bigskip
		\end{itemize}
		
		\item \textbf{rddensity}: discontinuity in density tests (manipulation testing) using both local polynomials and binomial tests.\medskip
		\begin{itemize}
			\item \texttt{rddensity}, \texttt{rdbwdensity}.\bigskip
		\end{itemize}
	
		\item \textbf{rdlocrand}: covariate balance, binomial tests, randomization inference methods (window selection \& inference).\medskip		
		\begin{itemize}
			\item \texttt{rdrandinf}, \texttt{rdwinselect}, \texttt{rdsensitivity}, \texttt{rdrbounds}.\bigskip
		\end{itemize}
		
		\item \textbf{rdmulti}: multiple cutoffs and multiple scores.\bigskip
		
		\item \textbf{rdpower}: power, sample selection and minimum detectable effect size.
		
	\end{itemize}
\end{frame}

\begin{frame}\frametitle{RD Plots}
	\begin{itemize}
		\item Main ingredients:\medskip
		\begin{itemize}
			\item Global smooth polynomial fit.\medskip
			\item Binned discontinuous local-means fit.\bigskip
		\end{itemize}
		
		\item Main goals:\medskip
		\begin{itemize}
			\item Graphical (heuristic) representation.\medskip
			\item Detention of discontinuities.\medskip
			\item Representation of variability.\bigskip
		\end{itemize}
		
		\item Tuning parameters:\medskip
		\begin{itemize}
			\item Global polynomial degree.\medskip
			\item Location (ES or QS) and number of bins.\bigskip
		\end{itemize}
		
		\item \textbf{Great to convey ideas but horrible to draw conclusions}.
	\end{itemize}
\end{frame}

\begin{frame}\frametitle{Estimation and Inference Methods}
	\begin{itemize}
		\item \textbf{Continuity/Extrapolation}: Local polynomial approach.\medskip
		\begin{itemize}
			\item Localization: bandwidth selection (trade-off bias and variance).\medskip
			\item Point estimation: ``flexible'' (nonparametric).\medskip
			\item Inference: robust bias-corrected methods.\bigskip
		\end{itemize}
		
		\item \textbf{Local Randomization}: finite-sample and large-sample inference.\medskip
		\begin{itemize}
			\item Localization: window selection (via local independence implications).\medskip
			\item Point estimation: parametric, finite-sample (Fisher) or large-sample (Neyman/SP).\medskip
			\item Inference: randomization inference (Fisher) or large-sample (Neyman/SP).\bigskip
		\end{itemize}
		
		\item Many refinements and other methods exist (EL, Bayesian, Uniformity, etc.).\medskip
		\begin{itemize}
			\item Do not offer much improvements in applications.\medskip
			\item Can be overly complicated (lack of transparency).\medskip
			\item Can depend on user-chosen tuning parameters (lack of replicability).\bigskip
		\end{itemize}
		
	\end{itemize}
\end{frame}

\begin{frame}\frametitle{Continuity/Extrapolation: Local Polynomial Methods} %% TO DO
	\begin{itemize}
		\item Global polynomial regression: \textbf{not recommended}.\medskip
		\begin{itemize}
			\item Runge's Phenomenon, counterintuitive weights, overfitting, lack of robustness.\bigskip
		\end{itemize}
		
		\item Local polynomial regression: captures idea of ``localization''.\medskip
	\end{itemize}
	\begin{center}
	\resizebox{.65\textwidth}{!}{\begin{tabular}{c}
		Choose low poly order ($p$) and weighting scheme ($K(\cdot)$) \vspace{0.15in}\\
		\arrowdown \vspace{0.15in}\\
		Choose bandwidth $h$: MSE-optimal or CE-optimal \vspace{0.15in}\\
		\arrowdown \vspace{0.15in}\\
		Construct point estimator $\hat{\tau}$\\ (MSE-optimal $h$ $\implies$ optimal estimator) \vspace{0.15in}\\
		\arrowdown \vspace{0.15in}\\
		Conduct robust bias-corrected inference\\
		(CE-optimal $h$ $\implies$ optimal distributional approximation)
	\end{tabular}}
	\end{center}
\end{frame}

\begin{frame}\frametitle{Local Polynomial Methods}
	\begin{itemize}
		\item \textbf{Idea}: approximate regression functions for control and treatment units \emph{locally}.\bigskip
		
		\item ``Local-linear'' ($p=1$) estimator (w/ weights $K(\cdot)$):\medskip
		\[%
		\begin{tabular}
			[c]{cc|cc}%
			$-h\leq X_{i}<\c:$ &  &  & $\c\leq X_{i}\leq h:$\\
			&  &  & \\
			$Y_{i}=\alpha_{-}+(X_{i}-\c)\cdot\beta_{-}+\varepsilon_{-,i}$ &  &  &
			$Y_{i}=\alpha_{+}+(X_{i}-\c)\cdot\beta_{+}+\varepsilon_{+,i}$%
		\end{tabular}
		\]
		\medskip
		
		\begin{itemize}
			\item Treatment effect (at the cutoff): $\hat{\tau}_{\mathtt{SRD}}(h)=\hat{\alpha}_{+}-\hat{\alpha}_{-}$\bigskip
		\end{itemize}
		
		\item Can be estimated using linear models (w/ weights $K(\cdot)$):
		\[Y_{i}=\alpha+\tau_{\mathtt{SRD}}\cdot T_{i}+(X_{i}-\c)\cdot\beta_{1}+T_{i}\cdot(X_{i}-\c)\cdot\gamma_{1}+\varepsilon_{i}, \qquad |X_{i}-c|\leq h\medskip\]
		
		\item Given $p$, $K$, $h$ chosen $\implies$ weighted least squares estimation.\smallskip
	\end{itemize}
\end{frame}

\begin{frame}
	\begin{figure}[h]
		\begin{center}
			\includegraphics[scale=0.60]{\pathfig/Fig6-Kernels-int1.pdf}
		\end{center}
	\end{figure}
\end{frame}

\begin{frame}
	\begin{figure}[h]
		\begin{center}
			\includegraphics[scale=0.60]{\pathfig/Fig5-LocalPolyEstimation-step1.pdf}
		\end{center}
	\end{figure}
\end{frame}

\begin{frame}
	\begin{figure}[h]
		\begin{center}
			\includegraphics[scale=0.60]{\pathfig/Fig5-LocalPolyEstimation-step2.pdf}
		\end{center}
	\end{figure}
\end{frame}

\begin{frame}
	\begin{figure}[h]
		\begin{center}
			\includegraphics[scale=0.60]{\pathfig/Fig5-LocalPolyEstimation-step3.pdf}
		\end{center}
	\end{figure}
\end{frame}

\begin{frame}
	\begin{figure}[h]
		\begin{center}
			\includegraphics[scale=0.60]{\pathfig/Fig5-LocalPolyEstimation-step4.pdf}
		\end{center}
	\end{figure}
\end{frame}

\begin{frame}
	\begin{figure}[h]
		\begin{center}
			\includegraphics[scale=0.60]{\pathfig/Fig5-LocalPolyEstimation-step5.pdf}
		\end{center}
	\end{figure}
\end{frame}

\begin{frame}
	\begin{figure}[h]
		\begin{center}
			\includegraphics[scale=0.60]{\pathfig/Fig5-LocalPolyEstimation-step6.pdf}
		\end{center}
	\end{figure}
\end{frame}

\begin{frame}
	\begin{figure}[h]
		\begin{center}
			\includegraphics[scale=0.60]{\pathfig/Fig5-LocalPolyEstimation-step7.pdf}
		\end{center}
	\end{figure}
\end{frame}

\begin{frame}
	\begin{figure}[h]
		\centering
		\includegraphics[scale=0.60]{\pathfig/Fig8-LinearApprox-step1.pdf}
	\end{figure}
\end{frame}

\begin{frame}
	\begin{figure}[h]
		\centering
		\includegraphics[scale=0.60]{\pathfig/Fig8-LinearApprox-step2.pdf}
	\end{figure}
\end{frame}

\begin{frame}\frametitle{Local Polynomial Methods: Choosing bandwidth ($p=1$)}
	\begin{itemize}
		\item Mean Square Error Optimal (MSE-optimal).
		\[h_{\mathtt{MSE}} = C_{\mathtt{MSE}}^{1/5}\cdot n^{-1/5}\qquad\qquad
		  C_{\mathtt{MSE}} = C(K)\cdot\frac{\mathsf{Var}(\hat{\tau}_{\mathtt{SRD}})}{\mathsf{Bias}(\hat{\tau}_{\mathtt{SRD}})^{2}}\]\smallskip
		
		\item Coverage Error Optimal (CE-optimal).
		\[h_{\mathtt{CE}} = C_{\mathtt{CE}}^{1/4}\cdot n^{-1/4}\qquad\qquad
		  C_{\mathtt{CE}} = C(K)\cdot\frac{\mathsf{Var}(\hat{\tau}_{\mathtt{SRD}})}{|\mathsf{Bias}(\hat{\tau}_{\mathtt{SRD}})|}\]\smallskip		
		
		\item \textbf{Key idea}:\medskip
		\begin{itemize}
			\item Trade-off bias and variance of $\hat{\tau}_{\mathtt{SRD}}(h)$. Heuristically:
			\[\uparrow~\mathsf{Bias}(\hat{\tau}_{\mathtt{SRD}})\qquad\Longrightarrow\qquad\downarrow\hat{h}
			  \text{\qquad and\qquad}
			  \uparrow~\mathsf{Var}(\hat{\tau}_{\mathtt{SRD}})\qquad\Longrightarrow\qquad\uparrow\hat{h}\]\smallskip
			
			\item Implementations: IK  first-generation while CCT second-generation plug-in rule.\newline
			      They differ in the way $\mathsf{Var}(\hat{\tau}_{\mathtt{SRD}})$ and $\mathsf{Bias}(\hat{\tau}_{\mathtt{SRD}})$ are estimated.\medskip
			
			\item Rule-of-thumb: $h_{\mathtt{CE}} \propto n^{1/20} \cdot h_{\mathtt{MSE}}$.
			
		\end{itemize}
		
		
	\end{itemize}
\end{frame}

\begin{frame}\frametitle{Conventional Inference Approach}
	\begin{itemize}
		\item ``Local-linear'' ($p=1$) estimator (w/ weights $K(\cdot)$):\medskip
		\[%
		\begin{tabular}
			[c]{cc|cc}%
			$-h\leq X_{i}<\c:$ &  &  & $\c\leq X_{i}\leq h:$\\
			&  &  & \\
			$Y_{i}=\alpha_{-}+(X_{i}-\c)\cdot\beta_{-}+\varepsilon_{-,i}$ &  &  &
			$Y_{i}=\alpha_{+}+(X_{i}-\c)\cdot\beta_{+}+\varepsilon_{+,i}$%
		\end{tabular}
		\]
		\medskip
		
		\begin{itemize}
			\item Treatment effect (at the cutoff): $\hat{\tau}_{\mathtt{SRD}}(h)=\hat{\alpha}_{+}-\hat{\alpha}_{-}$\bigskip
		\end{itemize}
	
		\item Construct usual t-test. For $\mathsf{H}_{0}:\tau_{\mathtt{SRD}}=0$,
		\[T(h)=\frac{\hat{\tau}_{\mathtt{SRD}}}{\sqrt{\mathsf{\hat{V}}}}
		  =\frac{\hat{\alpha}_{+}-\hat{\alpha}_{-}}{\sqrt{\mathsf{\hat{V}}_{+}+\mathsf{\hat{V}}_{-}}}\approx_{d}\mathcal{N}(0,1)
		\]
		\smallskip
		
		\item Na\"ive $95\%$ Confidence interval:
		\[I(h)=\Big[  ~\hat{\tau}_{\mathtt{SRD}}~\pm~1.96\cdot\sqrt{\mathsf{\hat{V}}} ~\Big]\]
	\end{itemize}
\end{frame}

\begin{frame}\frametitle{Robust Bias Correction Approach}
	\begin{itemize}
		\item \textbf{Key Problem}:
		\[T(h_{\mathtt{MSE}})=\frac{\hat{\tau}_{\mathtt{SRD}}}{\sqrt{\mathsf{\hat{V}}}}\approx_{d}\mathcal{N}(\mathsf{B},1)\quad\neq\quad\mathcal{N}(0,1)\]
		\begin{itemize}
			\item $\mathsf{B}$ captures bias due to misspecification error.\bigskip
		\end{itemize}
		
		\item \textbf{RBC distributional approximation}:
		\[T^{\mathtt{bc}}(h)
		  = \frac{\hat{\tau}_{\mathtt{SRD}}-\mathsf{\hat{B}}_{n}}{\sqrt{\mathsf{\hat{V}}}}
		  = \underset{\approx_{d}~\mathcal{N}(0,1)}{\underbrace{\frac{\hat{\tau}_{\mathtt{SRD}}-\mathsf{B}_{n}}{\sqrt{\mathsf{\hat{V}}}}}}
		    + \underset{\approx_{d}~\mathcal{N}(0,\gamma)}{\underbrace{\frac{\mathsf{B}-\mathsf{\hat{B}}}{\sqrt{\mathsf{\hat{V}}}}}}%
		\]
		\begin{itemize}
			\item $\mathsf{\hat{B}}$ is constructed to estimate leading bias $\mathsf{B}$, that is, misspecification error.\bigskip
		\end{itemize}

		\item \textbf{RBC $95\%$ Confidence Interval}:
		\[I_{\mathtt{RBC}} = \left[ ~\left(\hat{\tau}_{\mathtt{SRD}}-\color{blue} \mathsf{\hat{B}}\color{black}\right)
		~\pm~ 1.96\cdot\sqrt{\mathsf{\hat{V}} + \color{blue} \mathsf{\hat{W}}\color{black}} ~\right]
		\]		
	\end{itemize}
\end{frame}

\begin{frame}
	\begin{figure}[h]
		\centering\includegraphics[scale=0.6]{\pathfig/Fig7-EffectsBandwidths.pdf}
	\end{figure}
\end{frame}

\begin{frame}\frametitle{Empirical Illustration: Head Start (Ludwig and Miller, 2007,QJE)}
	\begin{itemize}
		\item \textbf{Problem}: impact of Head Start on Infant Mortality\bigskip
		
		\item \textbf{Data}:\medskip
		
		~\qquad$Y_{i}=$ child mortality 5 to 9 years old\medskip
		
		~\qquad$T_{i}=$ whether county received Head Start assistance\medskip
		
		~\qquad$X_{i}=$ 1960 poverty index\quad($\c=59.1984$)\medskip
		
		~\qquad$Z_{i}=$ see database.\bigskip
		
		\item \textbf{Potential outcomes}:\medskip
		
		~\qquad$Y_{i}(0)=$ child mortality if \textbf{had not received} Head
		Start\medskip
		
		~\qquad$Y_{i}(1)=$ child mortality if \textbf{had received} Head Start\bigskip
		
		\item \textbf{Causal Inference}:\medskip
		
		~\qquad$Y_{i}(0) \quad\neq\quad Y_{i}|T_{i}=0$\qquad\qquad and\qquad\qquad$Y_{i}(1) \quad\neq\quad Y_{i}|T_{i}=1$\bigskip
	\end{itemize}
\end{frame}

\hspace{-.3in}\begin{frame}\vspace{-.2in}\begin{figure}
		\includegraphics[scale=.7,angle=0] {\pathfig/RD-Ludwig-Miller-TableIII-landscape}
\end{figure}\end{frame}

\begin{frame}\frametitle{Local Randomization Framework}
	\begin{itemize}
		\item \textbf{Key idea}: treatment assignment as-if randomly assigned ``near'' cutoff.\newline
		There exists window $\W=[-w,w]$, with $-w<\c<w$, such that
		\[\text{for all }X_{i}\in \W\Longrightarrow T_{i}\text{ independent of }	(Y_{i}(0),Y_{i}(1))\]
		and possibly other conditions hold (e.g., knowledge of assignment mechanism).\medskip
		\begin{itemize}
			\item Conceptually different from continuity/extrapolation based methods.\medskip
			\item Challenge: window (neighborhood) selection.\medskip
			\item Challenge: small sample (estimation and) inference.\bigskip
		\end{itemize}
		
		\item \textbf{Two Steps} (analogous to local polynomial methods):\medskip
		\begin{itemize}
			\item Given window $\W$, (estimation and) inference is ``standard'': superpopulation, large-samples designed-based methods, randomization inference methods.\medskip
			\item Select windows $\W$ based on idea of local randomization.\medskip
			\item As-if randomly assigned assumption can be relaxed somewhat, but it is strong.\bigskip
		\end{itemize}
		
		\item \textbf{Catch}: as-if random assumption good approximation \textit{only very near cutoff}!\bigskip
		
	\end{itemize}
\end{frame}

\begin{frame}
	\[ T_i \text{ independent of } (Y_i(0),Y_i(1)) \text{ for all } X_i\in\W=[\c-w,c+w] \]
	\begin{figure}
		\vspace{-0.1in}
		\centering
		\includegraphics[scale=0.5]{\pathfig/Fig11-RDLocRand.pdf}
	\end{figure}
\end{frame}

\begin{frame}\frametitle{Local Randomization: Finite-sample and Large-sample Methods}
	\begin{itemize}
		\item Given $\W$ where local randomization holds:\medskip
		\begin{itemize}
		\item Randomization inference (Fisher): sharp null, finite-sample exact.\medskip
		\item Design-based (Neyman): large-sample valid, conservative.\medskip
		\item Large-sample standard: random potential outcomes, large-sample valid.\bigskip
		\end{itemize}
		
		\item All methods require window ($\W$) selection, and choice of statistic.\smallskip\newline
		      First two also require choice/assumptions assignment mechanism.\smallskip\newline
		      Covariate-adjustments (score or otherwise) possible.\bigskip
		
		\item $\W$ selection:\medskip
		\begin{itemize}
			\item Find neighborhood where (pre-intervention) covariate-balance holds.\medskip
			\item Find neighborhood where outcome and score independent.\medskip
			\item Domain-specific or application-specific choice.
		\end{itemize}
		
	\end{itemize}
\end{frame}

\begin{frame}
	\[ T_i \text{ independent of } (Y_i(0),Y_i(1)) \text{ for all } X_i\in\W=[\c-w,c+w] \]
	\begin{figure}
		\vspace{-0.1in}
		\centering
		\includegraphics[scale=0.5]{\pathfig/Fig11-RDLocRand.pdf}
	\end{figure}
\end{frame}

\begin{frame}
	\frametitle{Choosing $W_0$ using predetermined covariate $Z$}
	\begin{figure}
		\centering
		\includegraphics[scale=0.57]{\pathfig/Fig12-RDLocRand-WindowSel.pdf}
	\end{figure}
\end{frame}

\begin{frame}\frametitle{Empirical Illustration: Head Start (Ludwig and Miller, 2007,QJE)}
	\begin{itemize}
		\item \textbf{Problem}: impact of Head Start on Infant Mortality\bigskip
		
		\item \textbf{Data}:\medskip
		
		~\qquad$Y_{i}=$ child mortality 5 to 9 years old\medskip
		
		~\qquad$T_{i}=$ whether county received Head Start assistance\medskip
		
		~\qquad$X_{i}=$ 1960 poverty index\quad($\c=59.1984$)\medskip
		
		~\qquad$Z_{i}=$ see database.\bigskip
		
		\item \textbf{Potential outcomes}:\medskip
		
		~\qquad$Y_{i}(0)=$ child mortality if \textbf{had not received} Head
		Start\medskip
		
		~\qquad$Y_{i}(1)=$ child mortality if \textbf{had received} Head Start\bigskip
		
		\item \textbf{Causal Inference}:\medskip
		
		~\qquad$Y_{i}(0) \quad\neq\quad Y_{i}|T_{i}=0$\qquad\qquad and\qquad\qquad$Y_{i}(1) \quad\neq\quad Y_{i}|T_{i}=1$\bigskip
	\end{itemize}
\end{frame}

\hspace{-.3in}\begin{frame}\vspace{-.2in}\begin{figure}
		\includegraphics[scale=.7,angle=0] {\pathfig/RD-Ludwig-Miller-TableIII-landscape}
\end{figure}\end{frame}


%%%%%%%%%%%%%%%%%%%%%%%%%%%%%%%%%%%%%%%%%%%%%%%%%%%%%%%%%%%%
\section{Falsification and Validation} 
%%%%%%%%%%%%%%%%%%%%%%%%%%%%%%%%%%%%%%%%%%%%%%%%%%%%%%%%%%%%

\begin{frame}\frametitle{Falsification and Validation}
	\begin{itemize}
		\item \textbf{RD plots and related graphical methods}:\medskip		
		\begin{itemize}
			\item Always plot data: main advantage of RD designs. (Check if RD design!)\medskip	
			\item Plot histogram of $X_{i}$ (score) and its density. Careful: boundary bias.\medskip
			\item RD plot $\E[Y_{i}|X_{i}=x]$ (outcome) and $\E[Z_{i}|X_{i}=x]$ (pre-intervention covariates).\medskip
			\item Be careful not to oversmooth data/plots.\bigskip
		\end{itemize}
		
		\item \textbf{Sensitivity and related methods}:\medskip
		\begin{itemize}
			\item Score density continuity: binomial test and continuity test.\medskip
			\item Pre-intervention covariate no-effect (covariate balance).\medskip
			\item Placebo outcomes no-effect.\medskip
			\item Placebo cutoffs no-effect: informal continuity test away from $\c$.\medskip
			\item Donut hole: testing for outliers/leverage near $\c$.\medskip
			\item Different bandwidths: testing for misspecification error.\medskip
			\item Many other setting-specific (fuzzy, geographic, etc.).
		\end{itemize}
	\end{itemize}
\end{frame}

\begin{frame}\frametitle{Empirical Illustration: Head Start (Ludwig and Miller, 2007,QJE)}
	\begin{itemize}
		\item \textbf{Problem}: impact of Head Start on Infant Mortality\bigskip
		
		\item \textbf{Data}:\medskip
		
		~\qquad$Y_{i}=$ child mortality 5 to 9 years old\medskip
		
		~\qquad$T_{i}=$ whether county received Head Start assistance\medskip
		
		~\qquad$X_{i}=$ 1960 poverty index\quad($\c=59.1984$)\medskip
		
		~\qquad$Z_{i}=$ see database.\bigskip
		
		\item \textbf{Potential outcomes}:\medskip
		
		~\qquad$Y_{i}(0)=$ child mortality if \textbf{had not received} Head
		Start\medskip
		
		~\qquad$Y_{i}(1)=$ child mortality if \textbf{had received} Head Start\bigskip
		
		\item \textbf{Causal Inference}:\medskip
		
		~\qquad$Y_{i}(0) \quad\neq\quad Y_{i}|T_{i}=0$\qquad\qquad and\qquad\qquad$Y_{i}(1) \quad\neq\quad Y_{i}|T_{i}=1$\bigskip
	\end{itemize}
\end{frame}

%%%%%%%%%%%%%%%%%%%%%%%%%%%%%%%%%%%%%%%%%%%%%%%%%%%%%%%%%%%%
\section{Extrapolation and Other Topics (if time permits)}
%%%%%%%%%%%%%%%%%%%%%%%%%%%%%%%%%%%%%%%%%%%%%%%%%%%%%%%%%%%%

\begin{frame}\frametitle{RD Effects Away from the Cutoff}
	\begin{itemize}
		\item RD designs are credible, robust and easy to use.\medskip
		
		\item \textbf{Main drawback}: identification is ``local'' (not even causal if strict!).\medskip
		
		\item Ongoing research: How to extrapolate RD effects away from $\c$?\medskip		
		\begin{itemize}
			\item Internal vs. External validity.\bigskip
		\end{itemize}
		
		\item \textbf{Available methods}:\medskip
		\begin{itemize}
			\item Marginal effects: changes at the cutoff.\medskip
			\item Local randomization: effects near the cutoff.\medskip
			\item Covariate-adjustment local effects: selection-on-observables near the cutoff.\medskip
			\item Proxy variables: trace-out evolution of outcome away from cutoff.\medskip
			\item Setting specific: fuzzy RD, multi-cutoff RD, etc.
		\end{itemize}
	\end{itemize}
\end{frame}

\begin{frame}
	\begin{figure}[h]
		\centering\includegraphics[scale=0.6]{\pathfig/Fig4-Heterogeneity-step1.pdf}
	\end{figure}
\end{frame}

\begin{frame}
	\begin{figure}[h]
		\centering\includegraphics[scale=0.6]{\pathfig/Fig4-Heterogeneity-step2.pdf}
	\end{figure}
\end{frame}

\begin{frame}
	\begin{figure}[h]
		\centering\includegraphics[scale=0.6]{\pathfig/Fig4-Heterogeneity-step3.pdf}
	\end{figure}
\end{frame}

\begin{frame}

\begin{center}
	\LARGE Thank you!
\end{center}

\vspace{.5in}

\begin{center}
	\url{https://rdpackages.github.io/}\medskip
\end{center}

\end{frame}


\end{document}




























